\documentclass[a4paper,12pt,twoside]{article}
\usepackage[italian]{babel}
\usepackage[utf8]{inputenc}
\usepackage[T1]{fontenc}

\usepackage{nimbusmononarrow}

\renewcommand{\familydefault}{\sfdefault}
\usepackage[paper=a4paper,top=2cm,bottom=2cm,right=1cm,left=1cm]{geometry} % margini
\usepackage[colorlinks=false,
	allbordercolors={0.2 0.2 0.8},
	pdfborderstyle={/S/U/W 1}]{hyperref}
\usepackage{titlesec}
\usepackage{fancyhdr}
\usepackage{parskip}
%\usepackage{quattrocento}

\usepackage{colortbl}
\usepackage[usenames,dvipsnames]{xcolor} 
\usepackage{tcolorbox} 
\usepackage{tabularx}
\usepackage{array} 
\usepackage{colortbl} 
\usepackage{longtable}

\tcbuselibrary{skins}


 
  \tcbset{tab1/.style={enhanced,fonttitle=\bfseries\large,fontupper=\normalsize\sffamily, colback=blue!5!white,colbacktitle=Salmon!40!white, coltitle=black,center title,
  		overlay={
  			
  			{\begin{scope}[shift={([yshift=-5.15cm,xshift=19cm]frame.north west)},xscale=-1,yscale=-1] 
  					\path [draw=gray!50!black, fill=gray!50] {[rotate=-30]  (0,1/8) -- (1/12,1/8) arc (90:270:-1/32 and 1/8) -- (0,-1/8) -- (-1/8,-1/8-1/8)arc (270:90:1/16 and 1/4) -- (0,1/8) }  --cycle; 
  					\path [draw=gray!50!black,fill=none]{[rotate=-30]  (0,1/8) arc (90:270:1/32 and 1/8)};
  					\path [draw=gray!50!black,fill=none]{[rotate=150] (1/4,-1/6)  arc (270:90:-1/16 and 1/6)} ;
  					\path [draw=gray!50!black,fill=none]{[rotate=150] (1/3,-1/4)  arc (270:90:-1/10 and 1/4)} ;
  				\end{scope}
  				\begin{scope}[shift={([yshift=-5.15cm]frame.north west)},xscale=1,yscale=-1] 
  					\path [draw=gray!50!black, fill=gray!50] {[rotate=-30]  (0,1/8) -- (1/12,1/8) arc (90:270:-1/32 and 1/8) -- (0,-1/8) -- (-1/8,-1/8-1/8)arc (270:90:1/16 and 1/4) -- (0,1/8) }  --cycle; 
  					\path [draw=gray!50!black,fill=none]{[rotate=-30]  (0,1/8) arc (90:270:1/32 and 1/8)};
  					\path [draw=gray!50!black,fill=none]{[rotate=150] (1/4,-1/6)  arc (270:90:-1/16 and 1/6)} ;
  					\path [draw=gray!50!black,fill=none]{[rotate=150] (1/3,-1/4)  arc (270:90:-1/10 and 1/4)} ;
  				\end{scope}
  				\begin{scope}[shift={([yshift=0cm,xshift=0cm]frame.north west)},xscale=1,yscale=1] 
  					\path [draw=gray!50!black, fill=gray!50] {[rotate=-30]  (0,1/8) -- (1/12,1/8) arc (90:270:-1/32 and 1/8) -- (0,-1/8) -- (-1/8,-1/8-1/8)arc (270:90:1/16 and 1/4) -- (0,1/8) }  --cycle; 
  					\path [draw=gray!50!black,fill=none]{[rotate=-30]  (0,1/8) arc (90:270:1/32 and 1/8)};
  					\path [draw=gray!50!black,fill=none]{[rotate=150] (1/4,-1/6)  arc (270:90:-1/16 and 1/6)} ;
  					\path [draw=gray!50!black,fill=none]{[rotate=150] (1/3,-1/4)  arc (270:90:-1/10 and 1/4)} ;
  				\end{scope}
  				\begin{scope}[shift={([yshift=0cm,xshift=19cm]frame.north west)},xscale=-1,yscale=1] 
  					\path [draw=gray!50!black, fill=gray!50] {[rotate=-30]  (0,1/8) -- (1/12,1/8) arc (90:270:-1/32 and 1/8) -- (0,-1/8) -- (-1/8,-1/8-1/8)arc (270:90:1/16 and 1/4) -- (0,1/8) }  --cycle; 
  					\path [draw=gray!50!black,fill=none]{[rotate=-30]  (0,1/8) arc (90:270:1/32 and 1/8)};
  					\path [draw=gray!50!black,fill=none]{[rotate=150] (1/4,-1/6)  arc (270:90:-1/16 and 1/6)} ;
  					\path [draw=gray!50!black,fill=none]{[rotate=150] (1/3,-1/4)  arc (270:90:-1/10 and 1/4)} ;
  				\end{scope} }}}}
  	  
  	  
  	  
  	  
  	  
  	  \tcbset{tab2/.style={enhanced,fonttitle=\bfseries\large,fontupper=\normalsize\sffamily, colback=blue!5!white,colbacktitle=Salmon!40!white, coltitle=black,center title,
  	  		overlay={
  	  			
  	  			{\begin{scope}[shift={([yshift=-3.45cm,xshift=19cm]frame.north west)},xscale=-1,yscale=-1] 
  	  					\path [draw=gray!50!black, fill=gray!50] {[rotate=-30]  (0,1/8) -- (1/12,1/8) arc (90:270:-1/32 and 1/8) -- (0,-1/8) -- (-1/8,-1/8-1/8)arc (270:90:1/16 and 1/4) -- (0,1/8) }  --cycle; 
  	  					\path [draw=gray!50!black,fill=none]{[rotate=-30]  (0,1/8) arc (90:270:1/32 and 1/8)};
  	  					\path [draw=gray!50!black,fill=none]{[rotate=150] (1/4,-1/6)  arc (270:90:-1/16 and 1/6)} ;
  	  					\path [draw=gray!50!black,fill=none]{[rotate=150] (1/3,-1/4)  arc (270:90:-1/10 and 1/4)} ;
  	  				\end{scope}
  	  				\begin{scope}[shift={([yshift=-3.45cm]frame.north west)},xscale=1,yscale=-1] 
  	  					\path [draw=gray!50!black, fill=gray!50] {[rotate=-30]  (0,1/8) -- (1/12,1/8) arc (90:270:-1/32 and 1/8) -- (0,-1/8) -- (-1/8,-1/8-1/8)arc (270:90:1/16 and 1/4) -- (0,1/8) }  --cycle; 
  	  					\path [draw=gray!50!black,fill=none]{[rotate=-30]  (0,1/8) arc (90:270:1/32 and 1/8)};
  	  					\path [draw=gray!50!black,fill=none]{[rotate=150] (1/4,-1/6)  arc (270:90:-1/16 and 1/6)} ;
  	  					\path [draw=gray!50!black,fill=none]{[rotate=150] (1/3,-1/4)  arc (270:90:-1/10 and 1/4)} ;
  	  				\end{scope}
  	  				\begin{scope}[shift={([yshift=0cm,xshift=0cm]frame.north west)},xscale=1,yscale=1] 
  	  					\path [draw=gray!50!black, fill=gray!50] {[rotate=-30]  (0,1/8) -- (1/12,1/8) arc (90:270:-1/32 and 1/8) -- (0,-1/8) -- (-1/8,-1/8-1/8)arc (270:90:1/16 and 1/4) -- (0,1/8) }  --cycle; 
  	  					\path [draw=gray!50!black,fill=none]{[rotate=-30]  (0,1/8) arc (90:270:1/32 and 1/8)};
  	  					\path [draw=gray!50!black,fill=none]{[rotate=150] (1/4,-1/6)  arc (270:90:-1/16 and 1/6)} ;
  	  					\path [draw=gray!50!black,fill=none]{[rotate=150] (1/3,-1/4)  arc (270:90:-1/10 and 1/4)} ;
  	  				\end{scope}
  	  				\begin{scope}[shift={([yshift=0cm,xshift=19cm]frame.north west)},xscale=-1,yscale=1] 
  	  					\path [draw=gray!50!black, fill=gray!50] {[rotate=-30]  (0,1/8) -- (1/12,1/8) arc (90:270:-1/32 and 1/8) -- (0,-1/8) -- (-1/8,-1/8-1/8)arc (270:90:1/16 and 1/4) -- (0,1/8) }  --cycle; 
  	  					\path [draw=gray!50!black,fill=none]{[rotate=-30]  (0,1/8) arc (90:270:1/32 and 1/8)};
  	  					\path [draw=gray!50!black,fill=none]{[rotate=150] (1/4,-1/6)  arc (270:90:-1/16 and 1/6)} ;
  	  					\path [draw=gray!50!black,fill=none]{[rotate=150] (1/3,-1/4)  arc (270:90:-1/10 and 1/4)} ;
  	  \end{scope} }}}}
    
    
    
     \tcbset{tab3/.style={enhanced,fonttitle=\bfseries\large,fontupper=\normalsize\sffamily, colback=blue!5!white,colbacktitle=Salmon!40!white, coltitle=black,center title,
    		overlay={
    			
    			{\begin{scope}[shift={([yshift=-4.6cm,xshift=19cm]frame.north west)},xscale=-1,yscale=-1] 
    					\path [draw=gray!50!black, fill=gray!50] {[rotate=-30]  (0,1/8) -- (1/12,1/8) arc (90:270:-1/32 and 1/8) -- (0,-1/8) -- (-1/8,-1/8-1/8)arc (270:90:1/16 and 1/4) -- (0,1/8) }  --cycle; 
    					\path [draw=gray!50!black,fill=none]{[rotate=-30]  (0,1/8) arc (90:270:1/32 and 1/8)};
    					\path [draw=gray!50!black,fill=none]{[rotate=150] (1/4,-1/6)  arc (270:90:-1/16 and 1/6)} ;
    					\path [draw=gray!50!black,fill=none]{[rotate=150] (1/3,-1/4)  arc (270:90:-1/10 and 1/4)} ;
    				\end{scope}
    				\begin{scope}[shift={([yshift=-4.6cm]frame.north west)},xscale=1,yscale=-1] 
    					\path [draw=gray!50!black, fill=gray!50] {[rotate=-30]  (0,1/8) -- (1/12,1/8) arc (90:270:-1/32 and 1/8) -- (0,-1/8) -- (-1/8,-1/8-1/8)arc (270:90:1/16 and 1/4) -- (0,1/8) }  --cycle; 
    					\path [draw=gray!50!black,fill=none]{[rotate=-30]  (0,1/8) arc (90:270:1/32 and 1/8)};
    					\path [draw=gray!50!black,fill=none]{[rotate=150] (1/4,-1/6)  arc (270:90:-1/16 and 1/6)} ;
    					\path [draw=gray!50!black,fill=none]{[rotate=150] (1/3,-1/4)  arc (270:90:-1/10 and 1/4)} ;
    				\end{scope}
    				\begin{scope}[shift={([yshift=0cm,xshift=0cm]frame.north west)},xscale=1,yscale=1] 
    					\path [draw=gray!50!black, fill=gray!50] {[rotate=-30]  (0,1/8) -- (1/12,1/8) arc (90:270:-1/32 and 1/8) -- (0,-1/8) -- (-1/8,-1/8-1/8)arc (270:90:1/16 and 1/4) -- (0,1/8) }  --cycle; 
    					\path [draw=gray!50!black,fill=none]{[rotate=-30]  (0,1/8) arc (90:270:1/32 and 1/8)};
    					\path [draw=gray!50!black,fill=none]{[rotate=150] (1/4,-1/6)  arc (270:90:-1/16 and 1/6)} ;
    					\path [draw=gray!50!black,fill=none]{[rotate=150] (1/3,-1/4)  arc (270:90:-1/10 and 1/4)} ;
    				\end{scope}
    				\begin{scope}[shift={([yshift=0cm,xshift=19cm]frame.north west)},xscale=-1,yscale=1] 
    					\path [draw=gray!50!black, fill=gray!50] {[rotate=-30]  (0,1/8) -- (1/12,1/8) arc (90:270:-1/32 and 1/8) -- (0,-1/8) -- (-1/8,-1/8-1/8)arc (270:90:1/16 and 1/4) -- (0,1/8) }  --cycle; 
    					\path [draw=gray!50!black,fill=none]{[rotate=-30]  (0,1/8) arc (90:270:1/32 and 1/8)};
    					\path [draw=gray!50!black,fill=none]{[rotate=150] (1/4,-1/6)  arc (270:90:-1/16 and 1/6)} ;
    					\path [draw=gray!50!black,fill=none]{[rotate=150] (1/3,-1/4)  arc (270:90:-1/10 and 1/4)} ;
    \end{scope} }}}}


\tcbset{tab4/.style={enhanced,fonttitle=\bfseries\large,fontupper=\normalsize\sffamily, colback=blue!5!white,colbacktitle=Salmon!40!white, coltitle=black,center title,
		overlay={
			
			{\begin{scope}[shift={([yshift=-6.45cm,xshift=19cm]frame.north west)},xscale=-1,yscale=-1] 
					\path [draw=gray!50!black, fill=gray!50] {[rotate=-30]  (0,1/8) -- (1/12,1/8) arc (90:270:-1/32 and 1/8) -- (0,-1/8) -- (-1/8,-1/8-1/8)arc (270:90:1/16 and 1/4) -- (0,1/8) }  --cycle; 
					\path [draw=gray!50!black,fill=none]{[rotate=-30]  (0,1/8) arc (90:270:1/32 and 1/8)};
					\path [draw=gray!50!black,fill=none]{[rotate=150] (1/4,-1/6)  arc (270:90:-1/16 and 1/6)} ;
					\path [draw=gray!50!black,fill=none]{[rotate=150] (1/3,-1/4)  arc (270:90:-1/10 and 1/4)} ;
				\end{scope}
				\begin{scope}[shift={([yshift=-6.45cm]frame.north west)},xscale=1,yscale=-1] 
					\path [draw=gray!50!black, fill=gray!50] {[rotate=-30]  (0,1/8) -- (1/12,1/8) arc (90:270:-1/32 and 1/8) -- (0,-1/8) -- (-1/8,-1/8-1/8)arc (270:90:1/16 and 1/4) -- (0,1/8) }  --cycle; 
					\path [draw=gray!50!black,fill=none]{[rotate=-30]  (0,1/8) arc (90:270:1/32 and 1/8)};
					\path [draw=gray!50!black,fill=none]{[rotate=150] (1/4,-1/6)  arc (270:90:-1/16 and 1/6)} ;
					\path [draw=gray!50!black,fill=none]{[rotate=150] (1/3,-1/4)  arc (270:90:-1/10 and 1/4)} ;
				\end{scope}
				\begin{scope}[shift={([yshift=0cm,xshift=0cm]frame.north west)},xscale=1,yscale=1] 
					\path [draw=gray!50!black, fill=gray!50] {[rotate=-30]  (0,1/8) -- (1/12,1/8) arc (90:270:-1/32 and 1/8) -- (0,-1/8) -- (-1/8,-1/8-1/8)arc (270:90:1/16 and 1/4) -- (0,1/8) }  --cycle; 
					\path [draw=gray!50!black,fill=none]{[rotate=-30]  (0,1/8) arc (90:270:1/32 and 1/8)};
					\path [draw=gray!50!black,fill=none]{[rotate=150] (1/4,-1/6)  arc (270:90:-1/16 and 1/6)} ;
					\path [draw=gray!50!black,fill=none]{[rotate=150] (1/3,-1/4)  arc (270:90:-1/10 and 1/4)} ;
				\end{scope}
				\begin{scope}[shift={([yshift=0cm,xshift=19cm]frame.north west)},xscale=-1,yscale=1] 
					\path [draw=gray!50!black, fill=gray!50] {[rotate=-30]  (0,1/8) -- (1/12,1/8) arc (90:270:-1/32 and 1/8) -- (0,-1/8) -- (-1/8,-1/8-1/8)arc (270:90:1/16 and 1/4) -- (0,1/8) }  --cycle; 
					\path [draw=gray!50!black,fill=none]{[rotate=-30]  (0,1/8) arc (90:270:1/32 and 1/8)};
					\path [draw=gray!50!black,fill=none]{[rotate=150] (1/4,-1/6)  arc (270:90:-1/16 and 1/6)} ;
					\path [draw=gray!50!black,fill=none]{[rotate=150] (1/3,-1/4)  arc (270:90:-1/10 and 1/4)} ;
\end{scope} }}}}


\tcbset{tab5/.style={enhanced,fonttitle=\bfseries\large,fontupper=\normalsize\sffamily, colback=blue!5!white,colbacktitle=Salmon!40!white, coltitle=black,center title,
		overlay={
			
			{
				\begin{scope}[shift={([yshift=0cm,xshift=0cm]frame.north west)},xscale=1,yscale=1] 
					\path [draw=gray!50!black, fill=gray!50] {[rotate=-30]  (0,1/8) -- (1/12,1/8) arc (90:270:-1/32 and 1/8) -- (0,-1/8) -- (-1/8,-1/8-1/8)arc (270:90:1/16 and 1/4) -- (0,1/8) }  --cycle; 
					\path [draw=gray!50!black,fill=none]{[rotate=-30]  (0,1/8) arc (90:270:1/32 and 1/8)};
					\path [draw=gray!50!black,fill=none]{[rotate=150] (1/4,-1/6)  arc (270:90:-1/16 and 1/6)} ;
					\path [draw=gray!50!black,fill=none]{[rotate=150] (1/3,-1/4)  arc (270:90:-1/10 and 1/4)} ;
				\end{scope}
				\begin{scope}[shift={([yshift=0cm,xshift=19cm]frame.north west)},xscale=-1,yscale=1] 
					\path [draw=gray!50!black, fill=gray!50] {[rotate=-30]  (0,1/8) -- (1/12,1/8) arc (90:270:-1/32 and 1/8) -- (0,-1/8) -- (-1/8,-1/8-1/8)arc (270:90:1/16 and 1/4) -- (0,1/8) }  --cycle; 
					\path [draw=gray!50!black,fill=none]{[rotate=-30]  (0,1/8) arc (90:270:1/32 and 1/8)};
					\path [draw=gray!50!black,fill=none]{[rotate=150] (1/4,-1/6)  arc (270:90:-1/16 and 1/6)} ;
					\path [draw=gray!50!black,fill=none]{[rotate=150] (1/3,-1/4)  arc (270:90:-1/10 and 1/4)} ;
\end{scope} }}}}

\tcbset{tab6/.style={enhanced,fonttitle=\bfseries\large,fontupper=\normalsize\sffamily, colback=blue!5!white,colbacktitle=Salmon!40!white, coltitle=black,center title,
		overlay={
			
			{\begin{scope}[shift={([yshift=-3.95cm,xshift=19cm]frame.north west)},xscale=-1,yscale=-1] 
					\path [draw=gray!50!black, fill=gray!50] {[rotate=-30]  (0,1/8) -- (1/12,1/8) arc (90:270:-1/32 and 1/8) -- (0,-1/8) -- (-1/8,-1/8-1/8)arc (270:90:1/16 and 1/4) -- (0,1/8) }  --cycle; 
					\path [draw=gray!50!black,fill=none]{[rotate=-30]  (0,1/8) arc (90:270:1/32 and 1/8)};
					\path [draw=gray!50!black,fill=none]{[rotate=150] (1/4,-1/6)  arc (270:90:-1/16 and 1/6)} ;
					\path [draw=gray!50!black,fill=none]{[rotate=150] (1/3,-1/4)  arc (270:90:-1/10 and 1/4)} ;
				\end{scope}
				\begin{scope}[shift={([yshift=-3.95cm]frame.north west)},xscale=1,yscale=-1] 
					\path [draw=gray!50!black, fill=gray!50] {[rotate=-30]  (0,1/8) -- (1/12,1/8) arc (90:270:-1/32 and 1/8) -- (0,-1/8) -- (-1/8,-1/8-1/8)arc (270:90:1/16 and 1/4) -- (0,1/8) }  --cycle; 
					\path [draw=gray!50!black,fill=none]{[rotate=-30]  (0,1/8) arc (90:270:1/32 and 1/8)};
					\path [draw=gray!50!black,fill=none]{[rotate=150] (1/4,-1/6)  arc (270:90:-1/16 and 1/6)} ;
					\path [draw=gray!50!black,fill=none]{[rotate=150] (1/3,-1/4)  arc (270:90:-1/10 and 1/4)} ;
				\end{scope}
				 }}}}


\tcbset{tab8/.style={enhanced,fonttitle=\bfseries\large,fontupper=\normalsize\sffamily, colback=blue!5!white,colbacktitle=Salmon!40!white, coltitle=black,center title}}

\tcbset{tab9/.style={enhanced,fonttitle=\bfseries\large,fontupper=\normalsize\sffamily, colback=blue!5!white,colbacktitle=Salmon!40!white, coltitle=black,center title,
		overlay={
			
			{\begin{scope}[shift={([yshift=-20cm,xshift=19cm]frame.north west)},xscale=-1,yscale=-1] 
					\path [draw=gray!50!black, fill=gray!50] {[rotate=-30]  (0,1/8) -- (1/12,1/8) arc (90:270:-1/32 and 1/8) -- (0,-1/8) -- (-1/8,-1/8-1/8)arc (270:90:1/16 and 1/4) -- (0,1/8) }  --cycle; 
					\path [draw=gray!50!black,fill=none]{[rotate=-30]  (0,1/8) arc (90:270:1/32 and 1/8)};
					\path [draw=gray!50!black,fill=none]{[rotate=150] (1/4,-1/6)  arc (270:90:-1/16 and 1/6)} ;
					\path [draw=gray!50!black,fill=none]{[rotate=150] (1/3,-1/4)  arc (270:90:-1/10 and 1/4)} ;
				\end{scope}
				\begin{scope}[shift={([yshift=-20cm]frame.north west)},xscale=1,yscale=-1] 
					\path [draw=gray!50!black, fill=gray!50] {[rotate=-30]  (0,1/8) -- (1/12,1/8) arc (90:270:-1/32 and 1/8) -- (0,-1/8) -- (-1/8,-1/8-1/8)arc (270:90:1/16 and 1/4) -- (0,1/8) }  --cycle; 
					\path [draw=gray!50!black,fill=none]{[rotate=-30]  (0,1/8) arc (90:270:1/32 and 1/8)};
					\path [draw=gray!50!black,fill=none]{[rotate=150] (1/4,-1/6)  arc (270:90:-1/16 and 1/6)} ;
					\path [draw=gray!50!black,fill=none]{[rotate=150] (1/3,-1/4)  arc (270:90:-1/10 and 1/4)} ;
				\end{scope}
				 }}}}
			 
\tcbset{tab10/.style={enhanced,fonttitle=\bfseries\large,fontupper=\normalsize\sffamily, colback=blue!5!white,colbacktitle=Salmon!40!white, coltitle=black,center title,
		overlay={
			
			{\begin{scope}[shift={([yshift=-6.35cm,xshift=19cm]frame.north west)},xscale=-1,yscale=-1] 
					\path [draw=gray!50!black, fill=gray!50] {[rotate=-30]  (0,1/8) -- (1/12,1/8) arc (90:270:-1/32 and 1/8) -- (0,-1/8) -- (-1/8,-1/8-1/8)arc (270:90:1/16 and 1/4) -- (0,1/8) }  --cycle; 
					\path [draw=gray!50!black,fill=none]{[rotate=-30]  (0,1/8) arc (90:270:1/32 and 1/8)};
					\path [draw=gray!50!black,fill=none]{[rotate=150] (1/4,-1/6)  arc (270:90:-1/16 and 1/6)} ;
					\path [draw=gray!50!black,fill=none]{[rotate=150] (1/3,-1/4)  arc (270:90:-1/10 and 1/4)} ;
				\end{scope}
				\begin{scope}[shift={([yshift=-6.35cm]frame.north west)},xscale=1,yscale=-1] 
					\path [draw=gray!50!black, fill=gray!50] {[rotate=-30]  (0,1/8) -- (1/12,1/8) arc (90:270:-1/32 and 1/8) -- (0,-1/8) -- (-1/8,-1/8-1/8)arc (270:90:1/16 and 1/4) -- (0,1/8) }  --cycle; 
					\path [draw=gray!50!black,fill=none]{[rotate=-30]  (0,1/8) arc (90:270:1/32 and 1/8)};
					\path [draw=gray!50!black,fill=none]{[rotate=150] (1/4,-1/6)  arc (270:90:-1/16 and 1/6)} ;
					\path [draw=gray!50!black,fill=none]{[rotate=150] (1/3,-1/4)  arc (270:90:-1/10 and 1/4)} ;
				\end{scope}
}}}}

\tcbset{tab11/.style={enhanced,fonttitle=\bfseries\large,fontupper=\normalsize\sffamily, colback=blue!5!white,colbacktitle=Salmon!40!white, coltitle=black,center title,
		overlay={
			
			{\begin{scope}[shift={([yshift=-6.12cm,xshift=19cm]frame.north west)},xscale=-1,yscale=-1] 
					\path [draw=gray!50!black, fill=gray!50] {[rotate=-30]  (0,1/8) -- (1/12,1/8) arc (90:270:-1/32 and 1/8) -- (0,-1/8) -- (-1/8,-1/8-1/8)arc (270:90:1/16 and 1/4) -- (0,1/8) }  --cycle; 
					\path [draw=gray!50!black,fill=none]{[rotate=-30]  (0,1/8) arc (90:270:1/32 and 1/8)};
					\path [draw=gray!50!black,fill=none]{[rotate=150] (1/4,-1/6)  arc (270:90:-1/16 and 1/6)} ;
					\path [draw=gray!50!black,fill=none]{[rotate=150] (1/3,-1/4)  arc (270:90:-1/10 and 1/4)} ;
				\end{scope}
				\begin{scope}[shift={([yshift=-6.12cm]frame.north west)},xscale=1,yscale=-1] 
					\path [draw=gray!50!black, fill=gray!50] {[rotate=-30]  (0,1/8) -- (1/12,1/8) arc (90:270:-1/32 and 1/8) -- (0,-1/8) -- (-1/8,-1/8-1/8)arc (270:90:1/16 and 1/4) -- (0,1/8) }  --cycle; 
					\path [draw=gray!50!black,fill=none]{[rotate=-30]  (0,1/8) arc (90:270:1/32 and 1/8)};
					\path [draw=gray!50!black,fill=none]{[rotate=150] (1/4,-1/6)  arc (270:90:-1/16 and 1/6)} ;
					\path [draw=gray!50!black,fill=none]{[rotate=150] (1/3,-1/4)  arc (270:90:-1/10 and 1/4)} ;
				\end{scope}
				\begin{scope}[shift={([yshift=0cm,xshift=0cm]frame.north west)},xscale=1,yscale=1] 
					\path [draw=gray!50!black, fill=gray!50] {[rotate=-30]  (0,1/8) -- (1/12,1/8) arc (90:270:-1/32 and 1/8) -- (0,-1/8) -- (-1/8,-1/8-1/8)arc (270:90:1/16 and 1/4) -- (0,1/8) }  --cycle; 
					\path [draw=gray!50!black,fill=none]{[rotate=-30]  (0,1/8) arc (90:270:1/32 and 1/8)};
					\path [draw=gray!50!black,fill=none]{[rotate=150] (1/4,-1/6)  arc (270:90:-1/16 and 1/6)} ;
					\path [draw=gray!50!black,fill=none]{[rotate=150] (1/3,-1/4)  arc (270:90:-1/10 and 1/4)} ;
				\end{scope}
				\begin{scope}[shift={([yshift=0cm,xshift=19cm]frame.north west)},xscale=-1,yscale=1] 
					\path [draw=gray!50!black, fill=gray!50] {[rotate=-30]  (0,1/8) -- (1/12,1/8) arc (90:270:-1/32 and 1/8) -- (0,-1/8) -- (-1/8,-1/8-1/8)arc (270:90:1/16 and 1/4) -- (0,1/8) }  --cycle; 
					\path [draw=gray!50!black,fill=none]{[rotate=-30]  (0,1/8) arc (90:270:1/32 and 1/8)};
					\path [draw=gray!50!black,fill=none]{[rotate=150] (1/4,-1/6)  arc (270:90:-1/16 and 1/6)} ;
					\path [draw=gray!50!black,fill=none]{[rotate=150] (1/3,-1/4)  arc (270:90:-1/10 and 1/4)} ;
\end{scope} }}}}

\tcbset{tab12/.style={enhanced,fonttitle=\bfseries\large,fontupper=\normalsize\sffamily, colback=blue!5!white,colbacktitle=Salmon!40!white, coltitle=black,center title,
		overlay={
			
			{\begin{scope}[shift={([yshift=-4.73cm,xshift=19cm]frame.north west)},xscale=-1,yscale=-1] 
					\path [draw=gray!50!black, fill=gray!50] {[rotate=-30]  (0,1/8) -- (1/12,1/8) arc (90:270:-1/32 and 1/8) -- (0,-1/8) -- (-1/8,-1/8-1/8)arc (270:90:1/16 and 1/4) -- (0,1/8) }  --cycle; 
					\path [draw=gray!50!black,fill=none]{[rotate=-30]  (0,1/8) arc (90:270:1/32 and 1/8)};
					\path [draw=gray!50!black,fill=none]{[rotate=150] (1/4,-1/6)  arc (270:90:-1/16 and 1/6)} ;
					\path [draw=gray!50!black,fill=none]{[rotate=150] (1/3,-1/4)  arc (270:90:-1/10 and 1/4)} ;
				\end{scope}
				\begin{scope}[shift={([yshift=-4.73cm]frame.north west)},xscale=1,yscale=-1] 
					\path [draw=gray!50!black, fill=gray!50] {[rotate=-30]  (0,1/8) -- (1/12,1/8) arc (90:270:-1/32 and 1/8) -- (0,-1/8) -- (-1/8,-1/8-1/8)arc (270:90:1/16 and 1/4) -- (0,1/8) }  --cycle; 
					\path [draw=gray!50!black,fill=none]{[rotate=-30]  (0,1/8) arc (90:270:1/32 and 1/8)};
					\path [draw=gray!50!black,fill=none]{[rotate=150] (1/4,-1/6)  arc (270:90:-1/16 and 1/6)} ;
					\path [draw=gray!50!black,fill=none]{[rotate=150] (1/3,-1/4)  arc (270:90:-1/10 and 1/4)} ;
				\end{scope}
				\begin{scope}[shift={([yshift=0cm,xshift=0cm]frame.north west)},xscale=1,yscale=1] 
					\path [draw=gray!50!black, fill=gray!50] {[rotate=-30]  (0,1/8) -- (1/12,1/8) arc (90:270:-1/32 and 1/8) -- (0,-1/8) -- (-1/8,-1/8-1/8)arc (270:90:1/16 and 1/4) -- (0,1/8) }  --cycle; 
					\path [draw=gray!50!black,fill=none]{[rotate=-30]  (0,1/8) arc (90:270:1/32 and 1/8)};
					\path [draw=gray!50!black,fill=none]{[rotate=150] (1/4,-1/6)  arc (270:90:-1/16 and 1/6)} ;
					\path [draw=gray!50!black,fill=none]{[rotate=150] (1/3,-1/4)  arc (270:90:-1/10 and 1/4)} ;
				\end{scope}
				\begin{scope}[shift={([yshift=0cm,xshift=19cm]frame.north west)},xscale=-1,yscale=1] 
					\path [draw=gray!50!black, fill=gray!50] {[rotate=-30]  (0,1/8) -- (1/12,1/8) arc (90:270:-1/32 and 1/8) -- (0,-1/8) -- (-1/8,-1/8-1/8)arc (270:90:1/16 and 1/4) -- (0,1/8) }  --cycle; 
					\path [draw=gray!50!black,fill=none]{[rotate=-30]  (0,1/8) arc (90:270:1/32 and 1/8)};
					\path [draw=gray!50!black,fill=none]{[rotate=150] (1/4,-1/6)  arc (270:90:-1/16 and 1/6)} ;
					\path [draw=gray!50!black,fill=none]{[rotate=150] (1/3,-1/4)  arc (270:90:-1/10 and 1/4)} ;
\end{scope} }}}}

\tcbset{tab13/.style={enhanced,fonttitle=\bfseries\large,fontupper=\normalsize\sffamily, colback=blue!5!white,colbacktitle=Salmon!40!white, coltitle=black,center title,
		overlay={
			
			{\begin{scope}[shift={([yshift=-7.45cm,xshift=19cm]frame.north west)},xscale=-1,yscale=-1] 
					\path [draw=gray!50!black, fill=gray!50] {[rotate=-30]  (0,1/8) -- (1/12,1/8) arc (90:270:-1/32 and 1/8) -- (0,-1/8) -- (-1/8,-1/8-1/8)arc (270:90:1/16 and 1/4) -- (0,1/8) }  --cycle; 
					\path [draw=gray!50!black,fill=none]{[rotate=-30]  (0,1/8) arc (90:270:1/32 and 1/8)};
					\path [draw=gray!50!black,fill=none]{[rotate=150] (1/4,-1/6)  arc (270:90:-1/16 and 1/6)} ;
					\path [draw=gray!50!black,fill=none]{[rotate=150] (1/3,-1/4)  arc (270:90:-1/10 and 1/4)} ;
				\end{scope}
				\begin{scope}[shift={([yshift=-7.45cm]frame.north west)},xscale=1,yscale=-1] 
					\path [draw=gray!50!black, fill=gray!50] {[rotate=-30]  (0,1/8) -- (1/12,1/8) arc (90:270:-1/32 and 1/8) -- (0,-1/8) -- (-1/8,-1/8-1/8)arc (270:90:1/16 and 1/4) -- (0,1/8) }  --cycle; 
					\path [draw=gray!50!black,fill=none]{[rotate=-30]  (0,1/8) arc (90:270:1/32 and 1/8)};
					\path [draw=gray!50!black,fill=none]{[rotate=150] (1/4,-1/6)  arc (270:90:-1/16 and 1/6)} ;
					\path [draw=gray!50!black,fill=none]{[rotate=150] (1/3,-1/4)  arc (270:90:-1/10 and 1/4)} ;
				\end{scope}
				\begin{scope}[shift={([yshift=0cm,xshift=0cm]frame.north west)},xscale=1,yscale=1] 
					\path [draw=gray!50!black, fill=gray!50] {[rotate=-30]  (0,1/8) -- (1/12,1/8) arc (90:270:-1/32 and 1/8) -- (0,-1/8) -- (-1/8,-1/8-1/8)arc (270:90:1/16 and 1/4) -- (0,1/8) }  --cycle; 
					\path [draw=gray!50!black,fill=none]{[rotate=-30]  (0,1/8) arc (90:270:1/32 and 1/8)};
					\path [draw=gray!50!black,fill=none]{[rotate=150] (1/4,-1/6)  arc (270:90:-1/16 and 1/6)} ;
					\path [draw=gray!50!black,fill=none]{[rotate=150] (1/3,-1/4)  arc (270:90:-1/10 and 1/4)} ;
				\end{scope}
				\begin{scope}[shift={([yshift=0cm,xshift=19cm]frame.north west)},xscale=-1,yscale=1] 
					\path [draw=gray!50!black, fill=gray!50] {[rotate=-30]  (0,1/8) -- (1/12,1/8) arc (90:270:-1/32 and 1/8) -- (0,-1/8) -- (-1/8,-1/8-1/8)arc (270:90:1/16 and 1/4) -- (0,1/8) }  --cycle; 
					\path [draw=gray!50!black,fill=none]{[rotate=-30]  (0,1/8) arc (90:270:1/32 and 1/8)};
					\path [draw=gray!50!black,fill=none]{[rotate=150] (1/4,-1/6)  arc (270:90:-1/16 and 1/6)} ;
					\path [draw=gray!50!black,fill=none]{[rotate=150] (1/3,-1/4)  arc (270:90:-1/10 and 1/4)} ;
\end{scope} }}}}

\tcbset{tab14/.style={enhanced,fonttitle=\bfseries\large,fontupper=\normalsize\sffamily, colback=blue!5!white,colbacktitle=Salmon!40!white, coltitle=black,center title,
		overlay={
			
			{\begin{scope}[shift={([yshift=-3cm,xshift=19cm]frame.north west)},xscale=-1,yscale=-1] 
					\path [draw=gray!50!black, fill=gray!50] {[rotate=-30]  (0,1/8) -- (1/12,1/8) arc (90:270:-1/32 and 1/8) -- (0,-1/8) -- (-1/8,-1/8-1/8)arc (270:90:1/16 and 1/4) -- (0,1/8) }  --cycle; 
					\path [draw=gray!50!black,fill=none]{[rotate=-30]  (0,1/8) arc (90:270:1/32 and 1/8)};
					\path [draw=gray!50!black,fill=none]{[rotate=150] (1/4,-1/6)  arc (270:90:-1/16 and 1/6)} ;
					\path [draw=gray!50!black,fill=none]{[rotate=150] (1/3,-1/4)  arc (270:90:-1/10 and 1/4)} ;
				\end{scope}
				\begin{scope}[shift={([yshift=-3cm]frame.north west)},xscale=1,yscale=-1] 
					\path [draw=gray!50!black, fill=gray!50] {[rotate=-30]  (0,1/8) -- (1/12,1/8) arc (90:270:-1/32 and 1/8) -- (0,-1/8) -- (-1/8,-1/8-1/8)arc (270:90:1/16 and 1/4) -- (0,1/8) }  --cycle; 
					\path [draw=gray!50!black,fill=none]{[rotate=-30]  (0,1/8) arc (90:270:1/32 and 1/8)};
					\path [draw=gray!50!black,fill=none]{[rotate=150] (1/4,-1/6)  arc (270:90:-1/16 and 1/6)} ;
					\path [draw=gray!50!black,fill=none]{[rotate=150] (1/3,-1/4)  arc (270:90:-1/10 and 1/4)} ;
				\end{scope}
				\begin{scope}[shift={([yshift=0cm,xshift=0cm]frame.north west)},xscale=1,yscale=1] 
					\path [draw=gray!50!black, fill=gray!50] {[rotate=-30]  (0,1/8) -- (1/12,1/8) arc (90:270:-1/32 and 1/8) -- (0,-1/8) -- (-1/8,-1/8-1/8)arc (270:90:1/16 and 1/4) -- (0,1/8) }  --cycle; 
					\path [draw=gray!50!black,fill=none]{[rotate=-30]  (0,1/8) arc (90:270:1/32 and 1/8)};
					\path [draw=gray!50!black,fill=none]{[rotate=150] (1/4,-1/6)  arc (270:90:-1/16 and 1/6)} ;
					\path [draw=gray!50!black,fill=none]{[rotate=150] (1/3,-1/4)  arc (270:90:-1/10 and 1/4)} ;
				\end{scope}
				\begin{scope}[shift={([yshift=0cm,xshift=19cm]frame.north west)},xscale=-1,yscale=1] 
					\path [draw=gray!50!black, fill=gray!50] {[rotate=-30]  (0,1/8) -- (1/12,1/8) arc (90:270:-1/32 and 1/8) -- (0,-1/8) -- (-1/8,-1/8-1/8)arc (270:90:1/16 and 1/4) -- (0,1/8) }  --cycle; 
					\path [draw=gray!50!black,fill=none]{[rotate=-30]  (0,1/8) arc (90:270:1/32 and 1/8)};
					\path [draw=gray!50!black,fill=none]{[rotate=150] (1/4,-1/6)  arc (270:90:-1/16 and 1/6)} ;
					\path [draw=gray!50!black,fill=none]{[rotate=150] (1/3,-1/4)  arc (270:90:-1/10 and 1/4)} ;
\end{scope} }}}}

\usepackage{ wasysym }
\pagestyle{fancy}
\fancyhf{}
\lhead{Beep Codes}
\rhead{ \includegraphics[height=0.75cm]{weeeopen}}
\cfoot{< \textbf{\thepage} >}

\begin{document}
	\begin{titlepage}
		\vspace*{\stretch{1.0}}
		\begin{center}
			\Large \textbf{Beep Codes} 
		\end{center}
	\vspace*{\stretch{2.0}}
	\end{titlepage}	
	\section{HP and Compaq Desktop PCs-BIOS Beep Codes}
	
	


	
	
		{\centering
			

			
			I seguenti segnali acustici  possono verificarsi durante il ripristino, il flashing o l'aggiornamento del BIOS.
			

			
		\begin{tcolorbox}[tab1,tabularx={X||X}]
			\textbf{Beeps} & \textbf{Descrizione}  \\\hline\hline
			1 beep breve                                                                                                    & Unità floppy legacy o unità CD/DVD non rilevata             \\\hline
			2 beep brevi                                                                                                    & Dischetto floppy o CD non rilevato                                    \\\hline
			
			3 beep 
			brevi
			& Impossibile avviare il flashing
			\ (ad esempio quando manca un'unità o un'immagine del BIOS)                                                                                \\ 
			\hline
			4 beep brevi
			&     Flashing non riuscito (errore di checksum, immagine corrotta,ecc.)                                                                            \\ 
			\hline
			5 beep brevi
			&     Ripristino del BIOS riuscito                                                                            \\ 
			\hline
			2 beep brevi, 2 beep lunghi
			&     Ripristino del BIOS riuscito                                                                            \\
		
		\end{tcolorbox}


	

	%{\Large\bell} \textbf{NOTA:}
	 

	%I seguenti codici possono variare a seconda della versione del BIOS

	\begin{tcolorbox}[tab2,tabularx={X||X}]
		\textbf{Beeps} & \textbf{Descrizione}  \\\hline\hline
		1 segnale acustico breve e 1 segnale acustico lungo                                                                                                   & Problema di memoria             \\\hline
		2 suoni brevi e 1 segnale acustico lungo
		

		(si ripete 5 volte)                                                                                                    & Impossibile inizializzare video o scheda video necessaria ma non installata                               \\\hline
		
		3 beep 
		brevi e 1 beep lungo
		& Errore di configurazione della CPU o tipo di CPU non compatibile                                                                               \\ 
		
	\end{tcolorbox}

	}





	\subsection{Beep codes per American Megatrends, Inc (AMI) BIOS}
	{\centering

			

	%	{\Large\bell} \textbf{NOTA:}
		

	%	Qui non sono elencati tutti i tipi di segnale acustico. I segnali acustici potrebbero variare.

			
	\begin{tcolorbox}[tab3,tabularx={X||X}]
		\textbf{Beeps} & \textbf{Descrizione}  \\\hline\hline
		1          & Errore di temporizzazione della memoria             \\\hline
		2      & Errore di parità di memoria                               \\\hline
		
		3
		& Errore di lettura o scrittura della memoria                                                                              \\\hline
		
		4,5,6,7
		& Errori irreversibili relativi al funzionamento di componenti critici della scheda madre, ad esempio la CPU.
		\\\hline
		
		8
		& Memoria video
		\\
		
	\end{tcolorbox}
	La seguente tabella, elenca i codici acustici AMI che possono verificarsi quando si tenta di eseguire il flashing o il ripristino del BIOS. 
	

	

	%{\Large\bell} \textbf{NOTA:}
	

	%Qui non sono elencati tutti i tipi di segnale acustico. I segnali acustici potrebbero variare.

	
	\begin{tcolorbox}[tab4,tabularx={X||X}]
		\textbf{Beeps} & \textbf{Descrizione}  \\\hline\hline
		1 & Nessun supporto rilevato             \\\hline
		2  & File ROM non trovato nella directory principale                                    \\\hline
		
		3 
		& Inserire il volume del supporto successivo                                                                                \\ 
		\hline
		4 
		&     Programmazione Flash riuscita                                                                            \\ 
		\hline
		5 
		&     Errore durante la lettura del file                                                                           \\ 
		\hline
		7
		&      EPROM Flash non rilevato                                                                            \\
		\hline
		10
		&
		Errore durante la cancellazione della memoria flash
		\\
		\hline
		11
		&  	
		Errore di programmazione flash
		\\
		\hline
		12
		& Problema con la dimensione del file ROM
		\\
		\hline
		13
		& L'immagine ROM non corrisponde ai parametri BIOS
		\\
		
	\end{tcolorbox}



%{\Large\bell} \textbf{NOTA:}


%Qui non sono elencati tutti i tipi di segnale acustico. I segnali acustici potrebbero variare.


\begin{tcolorbox}[tab5,tabularx={X||X}]
	\textbf{Beeps} & \textbf{Descrizione}  \\\hline\hline
	1 & Errore di aggiornamento della DRAM. C'è un problema nella
	memoria di sistema o nella scheda madre.             \\\hline
	2  & Errore di parità di memoria. Il circuito di parità non funziona correttamente.                                   \\\hline
	
	3 
	& Errore 64K RAM di base. C'è un problema con il primo 64 KB di memoria di sistema.                                                                               \\ 
	\hline
	4 
	&     Il timer di sistema non è operativo. C'è un problema con il timer (s) che controlla le funzioni sulla scheda madre.                           \\ 
	\hline
	5 
	&     Fallimento del processore La CPU del sistema ha avuto esito negativo.                                                                           \\ 
	\hline
6
	&      Gate A20 / Errore controller tastiera. Il controller IC della tastiera non funziona, impedendo al gate A20 di passare dalla modalità di protezione del processore.                         \\
	\hline
	7
	&
	Errore di eccezione in modalità virtuale.
	\\
	\hline
	8
	&  	
	Errore di memoria video. Il BIOS non può scrivere nella memoria del buffer del frame sulla scheda video.
	\\
	\hline
	9
	& Errore di checksum della ROM. Il chip BIOS ROM sulla scheda madre è probabilmente difettoso.
	\\
	\hline
	10
	& Errore di checksum CMOS. Qualcosa sulla scheda madre
	\\
	
\end{tcolorbox}

\begin{tcolorbox}[tab6,tabularx={X||X}]
	
	11 & Cattiva memoria cache. Un errore nella memoria cache di livello 2.            \\\hline
	1 segnale acustico lungo, 2 brevi  & Errore nel sistema video.                                   \\\hline
	
	1 segnale acustico lungo, 3 brevi 
	& È stato rilevato un errore nella memoria sopra 64 KB.                                                                               \\ 
	\hline
	1 segnale acustico lungo, 8 brevi
	&     Fallimento test display                                                                           \\ 
	\hline
	Sirena a due toni
	&     Bassa velocità della ventola della CPU, problema del livello di tensione.                                                                          \\ 
	
\end{tcolorbox}
    }
	\subsection{Beep codes per Phoenix BIOS}
	

	
	{\centering
	Quattro serie di beep indicano un errore fatale, ovvero esiste un problema che impedisce l'avvio del computer. Ad esempio, potrebbero esserci: due segnali acustici,una breve pausa, un segnale acustico, una breve pausa, un segnale acustico, una breve pausa e poi tre segnali acustico; questo è rappresentato come 2-1-1-3 nella tabella qui sotto.
	

	%{\Large\bell} \textbf{NOTA:}
	
	%I seguenti codici possono variare a seconda della versione del BIOS

	\begin{tcolorbox}[tab5,tabularx={X||X||X}]
		\textbf{Beeps} & \textbf{Post Code} & \textbf{Descrizione}  \\\hline\hline
		Un segnale acustico breve & 0 & Processo POST completato correttamente - avvio normale             \\\hline
		1-1-1-1  &  & Codice bip non confermato. Riposizionare i chip RAM o sostituire i chip RAM come possibile soluzione.                                    \\
		\hline
		1-1-1-3  & 02h & Verifica della  modalità reale                                    \\\hline
		
		1-1-2-1 
		& 04h & Individuazione del tipo di CPU                                                                               \\ 
		\hline
		1-1-2-3 
		& 06h &    Inizializzazione l'hardware del sistema                                                                            \\ 
		\hline
		1-1-3-1 
		& 08h &     Inizializzazione dei registri del chipset con i valori POST  iniziali                                                                           \\ 
		\hline
		1-1-3-2
		& 09h &   Impostazione del flag POST                                                                             \\
		\hline
		1-1-3-3
		& 0Ah &
		Inizializzazione registri della CPU
		\\
		\hline
		1-1-4-1
		& 0Ch & Inizializzazione della cache sui valori POST iniziali
		\\
		\hline
		1-1-4-3
		& 0Eh & Inizializzazione dell' I / O
		\\
		\hline
		1-2-1-1 & 10h
		& Inizializzare Power Management
		\\
	\end{tcolorbox}

\begin{tcolorbox}[tab8,tabularx={X||X||X}]
1-2-1-2
& 11h & Caricare registri alternativi con valori POST iniziali
\\
\hline
1-2-1-3 & 12h
& Passaggio a UserPatch0
\\
\hline
1-2-2-1
& 14h & Inizializzazione controller tastiera
\\
\hline
1-2-2-3 & 16h
&  	
Checksum della ROM del BIOS
\\
\hline
1-2-3-1
& 18h & Inizializzazione del timer 8254
\\
\hline
1-2-3-3 & 1Ah
& Inizializzazione del controller DMA 8237
\\
\hline
1-2-4-1
& 1Ch & Reimpostazione del controller di interrupt programmabile
\\
\hline
1-3-1-1 & 20h
& Test aggiornamento DRAM
\\
\hline
1-3-1-3
& 22h & Test controller tastiera 8742
\\
\hline
1-3-2-1 & 24h
& Impostazione del registro del segmento ES su 4 GB
\\
\hline
1-3-3-1
& 28h & Dimensionamento automatico DRAM
\\
\hline
1-3-3-3 & 2Ah
& Cancellazione della RAM di base da 512 KB
\\
\hline
1-3-4-1
& 2Ch &  	
Verifica delle righe indirizzi di base 512
\\
\hline
1-3-4-3 & 2Eh
& Verifica della memoria di base da 512 KB
\\
\hline
1-4-1-3
& 32h & Test frequenza bus-clock CPU
\\
\hline
1-4-2-1 & 34h
&  	
Errore di lettura/scrittura della RAM del CMOS
\\
\hline
1-4-2-4
& 37h & Reinizializzare il chipset
\\
\hline
1-4-3-1 & 38h
& Shadowing della ROM del BIOS di sistema
\\
\hline
1-4-3-2
& 39h & Reinizializzazione la cache
\\
\hline
1-4-3-3
& 3Ah & Dimensionamento automatico della cache
\\
\hline
1-4-4-1
& 3Ch & Configurazione dei registri avanzati del chipset
\\
\hline
1-4-4-2
& 3Dh & Carica i registri alternativi con valori CMOS
\\
\hline
2-1-1-1
& 40h & Impostazione della velocità iniziale della CPU
\\
\hline
2-1-1-3
& 42hs &  	
Inizializzazione del vettore di interrupt
\\
\hline
2-1-2-1
& 44h &  	
Inizializzazione degli interrupt del BIOS
\\
\hline
2-1-2-3
& 46h & Verifica delle informazioni sul copyright della ROM
\\
\hline
2-1-2-4
& 47h & Inizializzazione del programma di gestione per le ROM delle opzioni PCI
\\

\end{tcolorbox}

\begin{tcolorbox}[tab8,tabularx={X||X||X}]
	2-1-3-1
	& 48h & Confronto della configurazione video e CMOS
	\\
	\hline
	2-1-3-2
	& 49h & Inizializzazione delle periferiche e del bus PCI
	\\
	\hline
	2-1-3-3
	& 4Ah &  	
	Inizializzazione di tutte le schede video nel sistema
	\\
	\hline
	2-1-4-1
	& 4Ch & Shadowing della ROM del BIOS video
	\\
	\hline
	2-1-4-3
	& 4Eh & Visualizzazione delle informazioni sul copyright
	\\
	\hline
	2-2-1-1
	& 50h & Visualizzazione del tipo e della velocità della CPU
	\\
	\hline
	2-2-1-3
	& 52h & Prova tastiera
	\\
	\hline
	2-2-2-1 & 54h
	& Impostazione del clic dei tasti, se abilitato
	\\
	\hline
	2-2-2-3
	& 56h & Attivazione tastiera
	\\
	\hline
	2-2-3-1 & 58h
	& Verifica degli interrupt imprevisti
	\\
	\hline
	2-2-3-3
	& 5Ah & Visualizzazione del messaggio "Premere F2 per accedere a CONFIGURAZIONE"
	\\
	\hline
	2-2-4-1 & 5Ch
	& Test RAM tra 512 e 640KB
	\\
	\hline
	2-3-1-1
	& 60h &  	
	Verifica della memoria estesa
	\\
	\hline
	2-3-1-3 & 62h
	& Verifica delle righe degli indirizzi della memoria estesa
	\\
	\hline
	2-3-2-1
	& 64h & Passaggio  a UserPatch1
	\\
	\hline
	2-3-2-3 & 66h
	& Configurazione dei registri avanzati della cache
	\\
	\hline
	2-3-3-1
	& 68h & Abilitazione delle cache esterna e della CPU
	\\
	\hline
	2-3-3-2 & 69h
	& Inizializzazione dell’handler SMI
	\\
	\hline
	2-3-3-3
	& 6Ah &  	
	Visualizzazione delle dimensioni della cache esterna
	\\
	\hline
	2-3-4-1 & 6Ch
	& Visualizzazione del messaggio di shadowing
	\\
	\hline
	2-3-4-3
	& 6Eh &  	
	Visualizzazione dei segmenti non eliminabili
	\\
	\hline
	2-4-1-1 & 70h
	&  	
	Visualizzazione dei messaggi di errore
	\\
	\hline
	2-4-1-3
	& 72h & Controllo della presenza di errori di configurazione
	\\
	\hline
	2-4-2-1 & 74h
	& Verifica dell’orologio in tempo reale
	\\
	\hline
	2-4-2-3
	& 76h & Controllo della presenza di errori della tastiera
	\\
	\hline
	2-4-4-1
	& 7Ch &  	
	Impostazione dei vettori di interrupt hardware
	\\
	
	
\end{tcolorbox}

\begin{tcolorbox}[tab8,tabularx={X||X||X}]
2-4-4-3
& 7Eh & Verifica del coprocessore, se presente
\\
\hline
3-1-1-1
& 80h & Disattivazione delle porte I/O sulla scheda
\\
\hline
3-1-1-3
& 82h & Rilevamento e installazione delle porte esterne RS232
\\
\hline
3-1-2-1
& 84h & Rilevamento e installazione delle porte parallele esterne
\\
\hline
3-1-2-3
& 86h & Reinizializzazione delle porte I/O sulla scheda
\\
\hline
3-1-3-1
& 88h & Inizializzazione dell’area dati del BIOS
\\
\hline
3-1-3-3
& 8Ah & Inizializzazione dell’area dati estesa del BIOS
\\
\hline
3-1-4-1
& 8Ch & Inizializzazione del controller del floppy disk
\\
3-2-1-1
& 90h & Inizializzazione del controller del disco rigido
\\
\hline
3-2-1-2
& 91h & Inizializzazione del controller del disco rigido sul bus locale
\\
\hline
3-2-1-3
& 92h & Passaggio a UserPatch2
\\
\hline
3-2-2-1
& 94h & Disattivazione della riga dell’indirizzo A20
\\
\hline
3-2-2-3
& 96h & Cancellazione del registro del segmento ES di grandi dimensioni
\\
\hline
3-2-3-1
& 98h & Ricerca di ROM opzionali
\\
\hline
3-2-3-3
& 9Ah & Shadowing delle ROM opzionali
\\
\hline
3-2-4-1
& 9Ch & Impostazione del risparmio di energia
\\
\hline
3-2-4-3
& 9Eh & Attivazione degli interrupt hardware
\\
\hline
3-3-1-1
& A0h & Impostazione dell’ora
\\
\hline
3-3-1-3
& A2h & Controllo del blocco dei tasti
\\
\hline
3-3-3-1
& A8h & Cancellazione del messaggio F2
\\
\hline
3-3-3-3
& Aah & Ricerca della pressione del tasto F2
\\
\hline
3-3-4-1
& AChP & Accedere a CONFIGURAZIONE
\\
\hline
3-3-4-3
& Aeh & Cancellazione del flag in-POST
\\
\hline
3-4-1-1
& B0h & Controllo della presenza di errori
\\
\hline
3-4-1-3
& B2h & POST eseguito. Preparazione al riavvio del sistema operativo.
\\
\hline
3-4-2-1
& B4h & Un bip
\\
\hline
3-4-2-3
& B6h & Verifica della password (facoltativa)
\\
\hline
3-4-3-1
& B8h & Cancellazione della tabella GDT
\\
\hline
3-4-4-1
& BCh & Cancellazione dei controlli di parità
\\


\end{tcolorbox}

\begin{tcolorbox}[tab9,tabularx={X||X||X}]
	3-4-4-3
	& Beh & Cancellazione dello schermo (facoltativa)
	\\
	\hline
	3-4-4-4
	& BFh &  	
	Controllo dei promemoria dei virus e di backup
	\\
	\hline
	4-1-1-1
	& C0h &  	
	Tentativo di riavviare con INT 19
	\\
	\hline
	4-2-1-1
	& D0h & Errore dell’handler di interrupt
	\\
	\hline
	4-2-1-3
	& D2h & Errore di interrupt sconosciuto
	\\
	\hline
	4-2-2-1
	& D4h & Errore di interrupt in sospeso
	\\
	\hline
	4-2-2-3
	& D6h & Errore di inizializzazione della ROM opzionale
	\\
	\hline
	4-2-3-1
	& D8h & Errore di chiusura del sistema
	\\
	\hline
	4-2-3-3
	& Dah & Spostamento dei blocchi estesi
	\\
	\hline
	4-2-4-1
	& DCh & Errore di chiusura del sistema
	\\
	\hline
	4-2-4-3
	& Deh & Errore del controller della tastiera
	\\
	\hline
	4-3-1-3
	& E2h & Inizializzazione del chipset
	\\
	4-3-1-4 & E3h &  	
	Inizializzazione del contatore di aggiornamento           \\\hline
	4-3-2-1  & E4h &  	
	Controllo del Forced Flash                                 \\\hline
	
	4-3-2-2 
	& E5h & Controllo dello stato hardware della ROM                                                                               \\ 
	\hline
	4-3-2-3 
	& E6h &    La ROM del BIOS funziona                                                                           \\ 
	\hline
	4-3-2-4
	& E7h &    Esecuzione di un test competo della RAM                                                                          \\ 
	\hline
	4-3-3-1
	& E8h &    Inizializzazione OEM                                                                           \\
	\hline
	4-3-3-2
	& E9h &
	 	
	Inizializzazione del controller di interrupt
	\\
	\hline
	4-3-3-3
	& Eah & Errore nel codice bootstrap
	\\
	\hline
	4-3-3-4
	& Ebh & Inizializzazione di tutti i vettori
	\\
	\hline
	4-3-4-1 & Ech
	&  	
	Riavvio del programma Flash
	\\
	\hline
	4-3-4-2 & Edh
	&  	
	Inizializzazione del dispositivo di avvio
	\\
	\hline
	4-3-4-3 & Eeh
	&  	
	La lettura del codice di avvio è corretta
	\\
	\hline
	Sirena a due toni & 
	&  	
	Bassa velocità della ventola della CPU, problema del livello di tensione.
	\\
	
\end{tcolorbox}

	

%{\Large\bell} \textbf{NOTA:}

%I seguenti codici possono variare a seconda della versione del BIOS

\begin{tcolorbox}[tab5,tabularx={X||X}]
	\textbf{Beeps} & \textbf{Descrizione e cosa controllare}  \\\hline\hline
	Un segnale acustico continuo &  Un problema con la memoria o il video.            \\\hline
	1-1-1-3  & CPU / scheda madre difettosa. Verifica la modalità reale.                                   \\
	\hline
	1-1-1-3  & Verifica della  modalità reale                                    \\
\end{tcolorbox}

\begin{tcolorbox}[tab8,tabularx={X||X}]
	
	1-1-2-1  &  CPU / scheda madre difettosa.            \\\hline
	1-1-2-3 & Scheda madre difettosa o uno dei suoi componenti.                                   \\
	\hline
	1-1-3-1  & Scheda madre difettosa o uno dei suoi componenti. Inizializza i registri del chipset con i valori POST iniziali.                                   \\
	\hline
	1-1-3-2  & Scheda madre difettosa o uno dei suoi componenti.                                  \\
	\hline
	1-1-3-1   & Scheda madre difettosa o uno dei suoi componenti. Inizializza i registri del chipset con i valori POST iniziali.                                   \\
	\hline
	1-1-3-2 & Scheda madre difettosa o uno dei suoi componenti.                                 \\
	\hline
	1-1-3-3  & Scheda madre difettosa o uno dei suoi componenti. Inizializza i registri della CPU.                                  \\
	\hline
1-1-3-4  & Errore nei primi 64 KB di memoria.                                    \\
	\hline
	1-1-4-1  & Errore di cache di livello 2.                                   \\
	\hline
	1-1-4-3  & Errore porta I / O.                                  \\
	\hline
	1-2-1-1  & Errore di gestione dell'alimentazione.                                \\
	\hline
	1-2-1-3  & Scheda madre difettosa o uno dei suoi componenti.                                  \\
	\hline
	1-2-2-1   & Errore del controller della tastiera.                                 \\
	\hline
	1-2-2-3  & Errore ROM del BIOS.                                 \\
	\hline
	1-2-3-1  & Errore del timer di sistema.                                 \\
	\hline
	1-2-3-3 & Errore DMA.                                 \\
	\hline
	1-2-4-1  & Errore del controller IRQ.                                 \\
	\hline
	1-3-1-1   & Errore di aggiornamento della DRAM.                                \\
	\hline
	1-3-1-3  & Guasto alla porta A20.                                 \\
	\hline
	1-3-2-1 & Scheda madre difettosa o uno dei suoi componenti.                                  \\
	\hline
	1-3-3-1  & Errore di memoria estesa.                                 \\
	\hline
	1-3-4-3 & Errore nel primo 1 MB di memoria di sistema.                                 \\
	\hline
	1-4-2-4  & Errore della CPU.                                  \\
	\hline
	2-1-4-1 & Errore shadow del BIOS ROM.                                 \\
	\hline
	1-4-3-3 & Errore di cache di livello 2.                                \\
	\hline
	2-1-1-1 & Scheda madre difettosa o uno dei suoi componenti.                                 \\
	\hline
	2-1-2-1 & Errore IRQ.                                 \\
	\hline
	2-1-2-3 & Errore ROM del BIOS.                                 \\
	\hline
	2-1-3-2 & Guasto della porta I / O.                                 \\
	\hline
	2-1-3-3 & Errore del sistema video.                               \\
	\hline
	2-1-2-1  & Errore IRQ.                               \\
	\hline
	2-1-2-3 & Errore ROM del BIOS.                                \\
	\hline
	2-1-2-4 & Guasto della porta I / O.                                 \\
	\hline
	2-2-1-1 & Errore della scheda video.                               \\
	\hline
	2-3-3-3   & Errore del controller della tastiera.                                \\
	\hline
	2-2-3-1  & Errore IRQ.                             \\
	\hline
	2-2-4-1  & Errore nel primo 1 MB di memoria di sistema.                               \\
	\hline
	2-3-3-3  & Errore di memoria estesa.                             \\
	\hline
	2-3-2-1  & Scheda madre difettosa o uno dei suoi componenti.                             \\
	\hline
	2-3-3-1 & Errore di cache di livello 2.                              \\
\end{tcolorbox}

\begin{tcolorbox}[tab8,tabularx={X||X}]
	
	2-3-4-3  &  Errore della scheda madre o della scheda video.           \\\hline
	2-4-1-1  & Errore della scheda madre o della scheda video.                                  \\
	\hline
	2-4-1-3  & Scheda madre difettosa o uno dei suoi componenti.                                  \\
	\hline
	2-4-2-1  & Errore RTC.                                 \\
	\hline
	2-4-2-3    & Errore del controller della tastiera.                                   \\
	\hline
	2-4-4-1 & Errore IRQ.                                \\
	\hline
	3-1-2-3   & Errore porta I / O.                                \\
	\hline
	3-1-3-3  & Scheda madre difettosa o uno dei suoi componenti.                                   \\
	\hline
	3-2-1-2 & Guasto dell'unità floppy o del disco rigido.                                 \\
	\hline
	3-2-1-3  & Scheda madre difettosa o uno dei suoi componenti.                                 \\
	\hline
	3-2-4-3  & Errore IRQ.                   \\
	\hline
	3-3-1-1   & Errore RTC.                                 \\
	\hline
	3-3-1-3  & Errore blocco tastiera.                                \\
	\hline
	3-3-3-3  & Scheda madre difettosa o uno dei suoi componenti.                                \\
	\hline
	3-4-4-4  & Scheda madre difettosa o uno dei suoi componenti.                                \\
	4-1-1-1  & Guasto dell'unità floppy o del disco rigido.                               \\
	\hline
	4-2-2-1   & Errore IRQ.                               \\
	\hline
	4-2-4-1   & Scheda madre difettosa o uno dei suoi componenti.                              \\
	\hline
	4-2-4-3  & Errore del controller della tastiera.                                \\
	\hline
	4-3-4-3  & Scheda madre difettosa o uno dei suoi componenti.                                 \\
	\hline
	4-3-3-4  & Errore IRQ.                                \\
	\hline
	4-3-4-2 & Guasto dell'unità floppy o del disco rigido.                                \\
	\hline
	1-1-2 & CPU / scheda madre difettosa.                                 \\
	\hline
	1-1-3  & Errore di lettura / scrittura della scheda madre / CMOS difettoso.                                 \\
	\hline
	1-1-4 & Errore di checksum BIOS / BIOS ROM errato.                             \\
	\hline
	1-2-1  & Il timer di sistema non è operativo. C'è un problema con il timer (s) che controlla le funzioni sulla scheda madre.                                \\
	\hline
	1-2-3 & Scheda madre difettosa / errore DMA.                               \\
	\hline
	1-3-1  & Errore di aggiornamento della memoria.                             \\
	\hline
	1-3-4  & Errore nei primi 64 KB di memoria.                             \\
	\hline
	1-4-1  & Guasto della linea indirizzo.                              \\
	\hline
	1-4-2  & Errore di RAM di parità.                            \\
	\hline
	1-4-3  & Errore nel timer.                                \\
	\hline
	1-4-4  & Guasto alla porta NMI.                              \\
	\hline
	2- -  & Qualsiasi combinazione di segnali acustici dopo 2 indica un errore nei primi 64 KB di memoria.                              \\
	\hline
	3-1-1    & Errore Master DMA.                               \\
	\hline
	3-1-2  & Errore DMA slave.                           \\
	\hline
	3-1-4  & Interrompere l'errore del controller.                              \\
	\hline
	3-2-4  & Errore del controller della tastiera.                           \\
	\hline
	3-3-2  & Errore CMOS.                             \\
	\hline
	3-3-4  & Errore della scheda video.                            \\
	\hline
	3-4-1  & Errore della scheda video.                            \\
\end{tcolorbox}

\begin{tcolorbox}[tab10,tabularx={X||X}]
3-1-1    & Errore Master DMA.                               \\
\hline
4-2-1   & Errore nel timer.                          \\
\hline
4-2-2   & Errore di arresto del CMOS.                              \\
\hline
4-2-3  & Errore Gate A20.                           \\
\hline
4-2-4  & Interruzione imprevista in modalità protetta.                            \\
\hline
4-3-1  & Errore nel test della RAM.                           \\
\hline
4-3-3  & Errore nel timer.                           \\
\hline
4-3-4  & Errore dell'orologio dell'ora del giorno.                          \\
\hline
4-4-1  & Errore della porta seriale.                          \\
\hline
4-4-2   & Guasto della porta parallela.                           \\
\hline
4-4-3   & Coprocessore matematico.                       \\
\end{tcolorbox}
    }
	

	

	

	
	\section{Beep codes per Award BIOS}
	{\centering
		

			

		%{\Large\bell} \textbf{NOTA:}
		
		%I seguenti codici possono variare a seconda della versione del BIOS.
		
		Se vengono rilevati altri problemi hardware correggibili, il BIOS visualizza un messaggio.
		

		
	\begin{tcolorbox}[tab11,tabularx={X||X}]
		\textbf{Beeps} & \textbf{Descrizione}  \\\hline\hline
		1 lungo, 2 brevi    & Indica che un errore video si è verificato e il BIOS non può inzializzare lo schermo video per visualizzare informazioni supplementari                            \\
		\hline
		1 lungo, 3 brevi   & Scheda video non rilevata (reinserire la scheda video) o scheda video difettosa                         \\
		\hline
		Beeps ripetuti  & Problemi di RAM                              \\
		\hline
		Segnale acustico ripetuto ad alta frequenza mentre il PC è in esecuzione & Surriscaldamento processore (CPU)                          \\
		\hline
		Beep alternati ripetuti ad alta e bassa frequenza  & Problema con il processore (CPU), eventualmente danneggiato.                            \\
	\end{tcolorbox}
}
\newpage
\section{Beep codes per Dell}





{\centering
	\begin{tcolorbox}[tab12,tabularx={X||X}]
		\textbf{Beeps} & \textbf{Descrizione}  \\\hline\hline
		1    & BIOS ROM corrotta o guasta                         \\
		\hline
		2  & Memoria ( RAM ) non rilevata.                       \\
		\hline
		3  & Scheda madre guasta                            \\
		\hline
		4 & Memoria ( RAM ) guasta                          \\
		\hline
		5  & Batteria CMOS guasta                           \\
		\hline
		6  & Scheda video guasta                            \\
		\hline
		7  & Processore (CPU) difettosa                            \\
	\end{tcolorbox}
}




\section{Beep codes per IBM BIOS}
{\centering
	

	


	%{\Large\bell} \textbf{NOTA:}
	
	%I seguenti codici possono variare a seconda della versione del BIOS.
	

	\begin{tcolorbox}[tab13,tabularx={X||X}]
		\textbf{Beeps} & \textbf{Descrizione}  \\\hline\hline
		Nessun segnale acustico    & Nessuna alimentazione                        \\
		\hline
		1 bip breve & POST normale, il computer è ok                       \\
		\hline
		2 brevi beep  & Errore POST, revisionare lo schermo per il codice errore                           \\
		\hline
		Beep continuo & No power, loose card, or short.                       \\
		\hline
		Brevi beep ripetuti & No power, loose card, or short.                          \\
		\hline
		1 lungo e 1 breve  & Problema alla scheda madre                            \\
		\hline
		1 lungo e 2 brevi  & Problema video (Mono/CGA circuiti di visualizzazione)                           \\
		\hline
		1 lungo e 3 brevi  & Video (EGA) circuiti di visualizzazione                          \\
		\hline
		3 lunghi  & Errore tastiera o scheda tastiera                          \\
		\hline
		1 segnale acustico, display vuoto o non corretto  & Video circuiti di visualizzazione                           \\
	\end{tcolorbox}
}





\section{Toni di avvio Macintosh}
{\centering
	

	


	%{\Large\bell} \textbf{NOTA:}
	
	%I seguenti codici possono variare a seconda della versione del BIOS.
	

	\begin{tcolorbox}[tab14,tabularx={X||X}]
		\textbf{Toni} & \textbf{Errore}  \\\hline\hline
		Tono di errore. (Due serie di diverse tonalità)    & Problemi con scheda logica o bus SCSI.                        \\
		\hline
		Tono di avvio, le rotazioni di unità, non il video & POST normale, il computer è ok                       \\
		\hline
		Acceso, nessun tono  & Problema della scheda logica                         \\
		\hline
		Tono alto, quattro toni più alti. & Problema con SIMM.                      \\
		
	\end{tcolorbox}
}

	\newpage
	\tableofcontents
	
\end{document}